\documentclass[aps,prl,amsmath,amssymb,groupedaddress,10pt,superscriptaddress,floatfix,twocolumn,showkeys,longbibliography]{revtex4-1} %\documentclass[pra,showpacs,amsfonts,preprint,amsmath,superscriptaddress,floatfix]{revtex4}

%\usepackage[german]{babel}
\usepackage{array,amsmath,amsfonts,amssymb,dsfont,tabularx,multirow}
%\usepackage{hhline}
%\usepackage[T1]{fontenc} \usepackage{ae} \usepackage{color} %\usepackage{times,mathptm}
\usepackage{graphicx} 
%\usepackage{soul} % added 09282017 YL
%\usepackage[dvipsnames]{xcolor}
%\DeclareMathOperator{\Tr}{Tr}
%\usepackage{amssymb}% http://ctan.org/pkg/amssymb
%\usepackage{pifont}% http://ctan.org/pkg/pifont
\newcommand{\cmark}{\ding{51}}%
\newcommand{\xmark}{\ding{55}}%
   
%\newcommand{\red}[1]{{\color{red}#1}}
%\newcommand{\cmt}[1]{{\color{blue}#1}}
%\newcommand{\mob}[1]{{\color{brown}#1}}
%\newcommand{\toth}[1]{{\color{red}#1}}
%\newcommand{\cire}[1]{{\color{blue}#1}}

\usepackage{makecell}
%\renewcommand\theadfont{\bfseries}

%\usepackage{gensymb} % added 06212017 YL

\begin{document}

\title{$\Lambda$-Enhanced loading of single $^{87} Rb$ atoms into optical dipole traps }

\author{M. O. Brown} \email{Mark.O.Brown@colorado.edu} 
\author{T.~Thiele}
\author{C. Kiehl}
\author{C. A. Regal} \affiliation{JILA, National Institute of Standards and Technology and University of Colorado, and
Department of Physics, University of Colorado, Boulder, Colorado 80309, USA}

\pacs{}

\renewcommand{\i}{{\mathrm i}} \def\1{\mathchoice{\rm 1\mskip-4.2mu l}{\rm 1\mskip-4.2mu l}{\rm
1\mskip-4.6mu l}{\rm 1\mskip-5.2mu l}} \newcommand{\ket}[1]{|#1\rangle} \newcommand{\tybra}[1]{\langle
#1|} \newcommand{\braket}[2]{\langle #1|#2\rangle} \newcommand{\kebtra}[2]{|#1\rangle\langle#2|}
\newcommand{\opelem}[3]{\langle #1|#2|#3\rangle} \newcommand{\projection}[1]{|#1\rangle\langle#1|}
\newcommand{\scalar}[1]{\langle #1|#1\rangle} \newcommand{\op}[1]{\hat{#1}}
\newcommand{\vect}[1]{\boldsymbol{#1}} \newcommand{\id}{\text{id}}

\begin{abstract}
We achieve enhanced loading of 89(1)\% of single $^{87} Rb$ atoms into shallow optical dipole traps by using lambda-enhanced grey-molasses to cool the atoms into the dipole traps and drive blue-detuned light-assisted collisions. The loading saturates at 0.6mK, but remains above 80\% at depths as low as 0.33 mK, an order of magnitude lower than what is required for other enhanced loading techniques. When applied to a large array of 10 by 10 dipole traps, an average loading of 80\% is retained. When further enhanced with real-time rearranging techniques, which we also demonstrate, this technique will allow for the creation of larger arrays uniformly filled with atoms.
\end{abstract}
\maketitle

Single atoms loaded into tightly confined optical dipole traps have proven to be a promising platform for quantum simulation using rydberg atoms or ground-state atoms.  However, procedures for loading atoms into the traps are probabilistic, and normal techniques reach loading efficiencies of only ~60\%. The atoms are typically cooled into the dipole traps using standard red-detuned pgc, and once the atoms are in close-proximity to each other, the red-detuned light excites the atoms into attractive molecular potentials, where the atoms gain a lot of energy and, usually, both atoms are heated out of the trap. Experiments are left with a single atom if an odd number of atoms entered the trap. The probabilistic nature of this loading has been a long-standing problem for experiments where the geometric configuration of the atoms is important, which is most of them. Several approaches have evolved toward alleviating this problem. 

One approach is to image the atoms after loading to learn the initial configuration, and use one of several schemes to rearrange the atoms into the desired configuration in real-time. This has been previously achieved in 1D using tweezers controlled by acousto-optical modulators, and in 2D and 3D using multiple sets of potentials for holding and moving the atoms, or by using state-dependent traps to accomplish the moves. However, this technique still wastes optical power by producing empty and unused dipole traps. The rearranging procedures, while impressive, have their own finite efficiency, take time to implement, and do not scale well. 

Alternatively, one can apply different loading techniques to increase the normally 60\% loading technique to upwards of 90\%. While clearly still limited, this greatly alleviates the probabilistic loading problem. However, previous techniques suffered from requiring 3mk deep traps, roughly a factor of 3 more than the normal loading technique. This is because these techniques use the same red-detuned polarization gradient-cooling to cool atoms into the dipole traps, and then use a blue-detuned beam to excite the atoms to blue-detuned repulsive molecular potentials once they are there. The repulse potentials allows more fine tuning of the amount of energy gained by the atoms during the collision, making it possible to give the atoms just-enough energy for only one atom to leave the trap. The high trap-depth causes light-shifts which suppress the red-detuned molecular transitions and the blue-detuned light can dominate. However, the high cost of needing 3x the trap power to achieve this loading rate has prevented its widespread adoption.

In this paper, we achieve enhanced loading into tightly-confined but shallow optical dipole traps by utilizing $\Lambda$-enhanced grey molasses to both cool the atoms into the dipole traps and to drive blue-detuned molecular transitions. Grey-Molasses is a velocity-selective coherent-population-trapping polarization gradient cooling technique. Two counter-propagating beams with different polarizations create a rapidly varying polarization-gradient in a vacuum with zero background field, allowing the polarization gradient to locally define the field. An atom exposed to this light in a given field will scatter the light until it reaches an equilibrium population of the different $m_F$ sublevels. However, if the atom is moving fast through the polarization gradient, the population within the sublevels will diabatically change. The atom is then not in the equilibrium, and will scatter light until it returns to the equilibrium. The orientation of fields and changes in population is such that in order reach equilibrium, it preferrentially scatters off of the laser beam that was travelling opposite to the atom. Thus, the atoms are slowed.  

In grey-molasses cooling, the lasers are chosen to be blue-detuned on a type-2 ($F' \le F$) transition. For such a transition, there are no cycling transitions, and so an atom populating a particular superposition of states will be dark to the light as the transitions from the different $m_f$ levels will interfere and cancel. In the lambda-enhanced scheme, the cooling and repumping beams are phase-coherent and in a lambda-scheme, making the dark states superpositions of both ground-state manifolds. This generally allows for colder cooling of the atoms, and importantly for our experiment, involves only blue-detuned light for the cooling, thereby avoiding any red-detuned molecular transitions to compete with.

In our experiment, we implement 3D $\Lambda$-enhanced grey molasses on the D1 line of Rb87 atoms. We use an EOM to generate the phase-coherent repump beam. In order to facilitate the optical access needed for the high-NA lens used to make the dipole traps, two beams are vertical and orthogonal, but the third beam which provides cooling along the high-na lens axis is at an obtuse angle. Because this beam travels through the windows of our cell at a sharp angle, of this geometry, the polarization of this beam is not pure and is not particularly well-balanced with the counter-propagating beam. This limits our free-space temperature to $30\mu K$, where other

\begin{itemize}
\item{[Grey-Molasses]}
    \begin{itemize}    
    \item{[Our Grey-Molasses setup.]}
        [D1 Transition of Rb87]. [EOM for repump]. [Beam Geometry]. [Free-space temperatures].
    \item{Fig: [Level Diagram]}
    \item{[Our Dipole Trap Setup]}
        [Tweezer generation system]. [High-NA Lens].  [Imaging System].  [Trap Depths, and variations between traps].
    \item{Fig: [Experiment Diagram, for both enhanced loading \& rearrangement stuffs]}
    
    \item{[Experiment Sequence]}
        [Load a MOT]. [Null Fields and turn off mot]. [Grey-Molasses Cooling]. [Image].
    
    \item{Figs: 2D-Scan Results}
        [1x1 Grey Molasses]. [10x10 grey molasses]. [1x1 Red PGC]. [Single lucky 10x10 image?]
    \end{itemize}
\item{[Rearranging]}
    \begin{itemize}
    \item{[Methods]}
    \item{[Results]}
    \end{itemize}
\item{[Conclusions]}

\end{itemize}

\section{Acknowledgments}

%\bibliography{regal_group_bib.bib}

\newpage

\appendix

\section{Supplementary information}
\label{supplementary}

\newpage
\end{document}






